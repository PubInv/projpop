\documentclass[11pt]{article}
\usepackage{geometry} % see geometry.pdf on how to lay out the page. There's lots.
\usepackage{hyperref}
\usepackage{graphicx}
\usepackage{gensymb}
\usepackage[affil-it]{authblk}
\usepackage[toc,page]{appendix}
\usepackage{pifont}
\usepackage{amsmath}
\usepackage{draftwatermark}

\SetWatermarkText{DRAFT}
\SetWatermarkScale{6}
\SetWatermarkLightness{0.95}

% \geometry{letter} % or letter or a5paper or ... etc
% \geometry{landscape} % rotated page geometry

% See the ``Article customise'' template for come common customisations

\title{Cheating Tsiolkovsky's Equation with Highly Unrealizable Physical Approaches}
\author{Robert L. Read
  \thanks{read.robert@gmail.com}
}
\affil{Founder, Public Invention, an educational non-profit.}


\date{\today}

%%% BEGIN DOCUMENT
\begin{document}

\maketitle

%% \tableofcontents

\section{Introduction}

Squirting a bunch of hot gases out of a rocket seems wasteful.
Tsiolkovsky's equation:

\[
\tag{The Rocket Equation} \Delta v = v_e \ln \frac{m_0}{m_f} 
\]

is so dreary in its demand that most of a rocket be devoted to fuel
that even its creator cheated it by inventing the multi-stage rocket.

If a rocket starts in free space, 
is it possible to build a rocket which is more efficient in power or
conservative of propellant reaction mass by designing the form
of the remaining propellant?

\section{Conservation of Linear Momentum}

Although it is promising to consider powering a rocket with a beam
from a base station, we may be able to learn something interesting from
considering a rocket starting in free space. After all, the base station
itself or the planet it rests upon is simply a large object in free space.

A chemical rocket in free space without gravity is a system that can't
change its center of mass. As the rocket fires, the system spreads out
in space, with the part be call the propellant moving in one direction
and the rocket moving in the other.


Since we are primarily interested in going in one direction fast, we
can think of this as a one-dimensional problem. The fact that a cloud
of gas spreads a little from our single axis is an unfortunate inefficiency.

We choose coordinates so that the directions we wish to go is the positive $x$ axis.
The origin is at the center of mass of our starting machine, which includes the rocket
and its propellant.
Conservation of linear moment applies to whatever form the propellant takes.
If $\dot{P_x}$ is the velocity of the center of mass of the propellant (which will
likely be negative), and $\dot{R_x}$ is the velocity of the rocket:

\[
\label{conslinmom} \tag{CLM} \dot{R_x}\cdot M_R = - \dot{P_x} \cdot M_P
\]

What would happen if instead of throwing gas molecules behind ourselves
we threw solid objects? In other words, what if we used a ``mass driver'',
without specifying how it is powered, to throw a solid propellant?
Could we change Tsiolkovsky's equation in some favorable way?

\section{Extruded Forms}

Suppose that our rocket could extrude an object extremely quickly.
If we extruded an object in the opposite direction we wish to travel,
by the conservation of linear momentum our rocket will move forward.

\[
\tag{Linear Extrusion} R_x - P_x  = \frac{s t}{2}
\]


After doing a lot of math, we determine that the speed of the rocket
at time $t$, assume an extrusion rate of $s$ in $m /s$ and density
of extrusion of $k$ in $kg / $ linear meter, 

\[
\dot{R_x} = \frac{k s^2 t}{2 m_0}
\]

where $m_0$ is the total inital mass of the rocket and the propellant:

\[
m_0 = m_R + m_P
\]

or, in terms or Tsiolkovsky's symols, $m_0 = T_m$ and $m_f = m_R$.

By solving for the time it takes to use up all of the propellant and
then substituting back into the form of the Tsiolkovsky equation:

\[
v_r = \frac{s (m_0-m_f)}{2 m_0}
\]

If we graph this against the Tsiolkovsky equation:

INSERT GRAPH HERE

we find that extruding a cylinder very quickly is not quite as
good as squirting gases at the same velocity behind us.
My interpretation of this is that being attached to the extruded
rod during until the momenet of separation holds the rocket back,
where as the acceleartion provided by each molecule of gas
allows the rocket to move forward without being attached.

\subsection{Develop Mathematics of Instantaneous Ejection and Relate}

\section{Constant Acceleration of Cylinder}

Suppose that we could extrude the rod not at a constant speed, but
at a constant acceleration? This is physically hard to engineer,
because as the mass of the rod behind us increases, the force required
to sustain a constant acceleration $A$ increases. Nonetheless if we
work out the math, we obtain:

\[
\tag{Constant Acceleration} R_x - P_x  = \frac{A t^2}{2}
\]

\label{Constant Acceleration} can be differentiated:

\[
\tag{Velocities under C.A.} \dot{R_x} - \dot{P_x}  = A \cdot t
\]

and the mass of the propellant will be:

\[
 M_P = (R_x - P_x)\cdot K \cdot 2
\]

or

\[
 M_P = A t^2 \cdot K 
\]


Rearranging and using \eqref{conslinmom}, we obtain:

\[
\dot{R_x}(1 + \frac{M_R}{M_P} ) = A \cdot t
\]
\[
=
\]
\[
\dot{R_x}(\frac{M_P + M_R}{M_P} ) = A \cdot t
\]
\[
=
\]
\[
\dot{R_x} = A \cdot t \cdot (\frac{M_P}{M_P + M_R} )
\]
\[
=
\]
\[
\dot{R_x} = A \cdot t \cdot (\frac{M_P}{m_0} )
\]
\[
=
\]
\[
\dot{R_x} = A \cdot t \cdot (\frac{A t^2 \cdot K }{m_0} )
\]
\[
=
\]
\[
\dot{R_x} = \frac{A^2 t^3 K}{m_0}
\]

...where A is the acceleration we support. If we again solve for the
expulsion of all of the propellant, we obtain:

\[
v_r = \frac{(m_0 - m_f)^\frac{3}{2}\sqrt{\frac{A}{K}}}{m_0}
\]

This goes up as $A$ goes up, and goes up as we dedicated more mass to the propellant as we would expect.
Note that it also goes up as $K$ the denisty of our extrusion goes down, which is perhaps interesting:
the thinnner a rod we can extrude the more efficient our system is.

NEED TO compare to Tsiolkovsky.

But of course this is unrealistic in that we cannot have infinite power in our rocket.

\section{Relate to A Plume of Bouncing Machines}

An alternative to producing a solid plume as a single unit is to eject discrete objects.
These objects can interact by bouncing off each other or potentially accelerating objects
in one direction or the other.  The conservation of linear momentum remains ironclad,
but we may now think of our rocket/propellant complex as a complicated machine which
is working to spread itself out in space. Hopefully we are working to send as much
mass behind us as quickly as possible.

Possibly our propellant could even be complete machines with their own power supply.
Thuse our rocket becomes a line of mass drivers spread out in space that may drive
objects with mass towards or away from the rocket.

The fact that it might be a little tricky to survice collision with or the ``catching'' and
``rethrowing'' of these objects we will ignore for the time being.

We assert that just as the Tsiolkovsky equation can be cheated by using a multi-stage
rocket, it can also be cheated by bouncing objects around in a line, reusing reaction mass.
This can potentially result in final propellant shape more efficient than a plume of
gas flying through and slowly spreading through space.

Even better, if one line of the mass drivers is anchored on a planet, we can
imagine a system in which the rocket is powered from the planet, where presumably power
is cheap, via the transfer or momentum to the rocket via objects driven to and fro from
the rocket to the planet.

\section{TODO}

Fill out the math in a reasonable way.  Fill in graphs. Do dimensional analysis.




\end{document}
